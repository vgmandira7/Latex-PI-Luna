Neste tópico serão apresentados projetos, que foram analisados, e possuem algumas semelhanças com esta proposta.

O estudo desenvolvido por \cite{Santos2020}, tem como objetivo analisar os impactos em uma plataforma de jogos educacionais para crianças com TDAH, com objetivo de desenvolver suas habilidades.  O sistema funciona como uma aplicação para smartphones, utilizando jogos e quebra-cabeças que estimulam a memória, a linguagem e o pensamento. Além disso, ao final do treinamento o aplicativo apresenta o desempenho dos alunos, para a visualização da sua evolução. Um questionário de aceitação sobre os impactos do software foi aplicado em duas escolas municipais, para estudar o desenvolvimento da plataforma. Com o levantamento dos dados, pôde-se observar que em meio à dificuldade dos alunos em assistir as aulas remotas de forma tradicional, o software auxiliou de forma positiva os alunos com TDAH a entender o conteúdo da aula. Ademais, o aplicativo demonstrou inúmeras satisfações em relação à melhora dos estudantes nas aulas, ao evoluírem sua concentração e aprenderem com um conteúdo desenvolvido exclusivamente para eles. Aliado a isso, a aceitação dos docentes foi positiva, despertando a curiosidade e atenção deles. No entanto, alguns pontos negativos foram observados, o alto consumo de memória que dificulta a instalação, além disso, certos níveis do software são pagos, o que ocasiona no bloqueio de algumas camadas, fazendo com que os alunos repitam exercícios e fiquem sem subir de nível, deixando o conteúdo repetitivo. Portanto, os resultados finais obtidos indicaram que o aplicativo é uma tecnologia de influência positiva na vida escolar dos acadêmicos, uma vez que ele ajuda em cada dificuldade, auxiliando na evolução educacional de cada criança.


O projeto realizado por \cite{Lauriano2024}, visa desenvolver uma aplicação baseada em Inteligência Artificial (IA), para auxiliar profissionais da área da saúde no  pré-diagnóstico do Transtorno do Espectro Autista (TEA) em crianças de 0 a 2 anos no Vale do Ribeira. O software foi desenvolvido para operar em dispositivos móveis, o qual utiliza o MultiLayer Perceptron (MLP), um modelo de rede neural, que coleta dados inseridos por usuários no protocolo Q-CHAT-10, o app analisa os dados fornecidos, extraindo suas principais características, para obter o pré-diagnóstico. Para obter os resultados sobre a eficácia e usabilidade do software, o estudo coletou 56 registros, dos quais 27 registros eram de crianças que não possuem TEA e 27 crianças com TEA (grupo de controle). Os usuários responderam um questionário com perguntas que identificam traços comportamentais do distúrbio, os resultados obtidos mostraram que a aplicação possui uma precisão de 90,7\% na classificação correta de crianças com ou sem TEA, a sensibilidade foi classificada com 92,6\% evidenciando a capacidade de identificação correta de crianças com TEA e a base de dados do projeto contou com 1.054 instâncias e o modelo foi treinado com validação cruzada K-FOLD de 10 dobras, com uma acurácia de aproximadamente 90\%. O estudo concluiu que o software desenvolvido tem potencial para o pré-diagnóstico de TEA, principalmente em regiões difíceis, ajudando na acessibilidade na identificação precoce do distúrbio. Porém, há limitações como: a necessidade de utilização em meios clínicos e um número maior de amostras, além disso, o uso de hardwares mais avançados, para a melhora do aplicativo. 

O último projeto, conduzido por \cite{Moller2024}, tem a finalidade de testar o uso de uma tecnologia de ensino a distância  e analisar seu impacto na velocidade do aprendizado que usa IA generativa para melhorar e acelerar o aprendizado por personalização. A aplicação consiste em uma assistente, a qual utiliza IA generativa para guardar e analisar o conteúdo e oferece diversas ferramentas para os estudos, especialmente o recurso de treinamento para provas e avaliações escolares, que cria exercícios sobre a matéria e devolve ao aluno um feedback corretivo. Para calcular o progresso, foi feito um estudo de coleta de dados no nível mensal com alunos que utilizam o software pelo menos uma vez (grupo de tratamento) e alunos que utilizam de um programa de ensino a distância (grupo de controle), esse teste foi feito com base no recurso de treinamento de exames do software, calculando o número de exames de cada aluno e o mês dentro do período relevante que usaram o aplicativo e o cálculo da média do grupo de controle. Os resultados foram analisados e comparados com os meses anteriores e posteriores do lançamento do software, os alunos que começaram a utilizar a aplicação já eram mais rápidos que os outros alunos, 24,1\%, contudo o uso da aplicação melhorou o desempenho desses alunos para 69.9\%. O uso dessa tecnologia aumentou em 46\% o progresso de estudo dos alunos em relação ao grupo de controle.

Com base nos artigos apresentados, o projeto atual demonstra maior semelhança com o desenvolvido por \cite{Moller2024}, ao empregar a IAG na personalização do conteúdo com o intuito de aprimorar o aprendizado e o desempenho dos alunos. Entretanto, enquanto \cite{Moller2024}. utilizam a IAG para adaptar o ritmo e o conteúdo, o presente projeto propõe uma abordagem inovadora ao personalizar o formato e o ambiente de aprendizagem de forma lúdica, considerando um traço neurocomportamental específico, o hiperfoco, em crianças com TDAH, o que constitui um diferencial metodológico relevante. Assim como na pesquisa de \cite{Santos2020}, o presente projeto também busca analisar o impacto de um software personalizado no processo de ensino-aprendizagem de crianças com TDAH, explorando recursos lúdicos e interativos para manter sua atenção e favorecer o desenvolvimento educacional. Já o estudo de \cite{Lauriano2024}, apresenta similaridade quanto ao uso de IA voltada à educação e ao apoio a distúrbios infantis, porém enquanto \cite{Lauriano2024} utilizam IA tradicional para classificação e predição, o projeto atual se apoia em IAG para criação e transformação de conteúdo, exigindo modelos e abordagens computacionais distintos, além disso, o foco permanece no aprimoramento da aprendizagem de crianças com TDAH, e não no diagnóstico precoce do TEA.
