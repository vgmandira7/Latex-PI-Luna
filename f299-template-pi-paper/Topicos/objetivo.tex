
O objetivo geral deste trabalho é desenvolver e validar uma metodologia de personalização lúdica de planos de aula para o Ensino Fundamental I, utilizando IAG como ferramenta para transformar conteúdos didáticos em estímulos de hiperfoco, com o propósito de aumentar a atenção e a retenção de aprendizagem em crianças de 6 a 10 anos com diagnóstico de TDAH. Para atingi-lo foram definidos os seguintes objetivos específicos:\\


\begin{itemize}
    \item Elaborar uma base de treinamento com pelo menos 100 planos de aula para um modelo de IAG, de modo que este identifique 80\% de precisão em oportunidades de personalização nos conteúdos.\\
    \item Desenvolver um ambiente interativo e gamificado que proporcione um aumento de pelo menos 25\% no tempo de interação contínua das crianças e um aumento de 15\% na performance em testes inseridos no aplicativo, em comparação aos métodos de ensino tradicionais.\\
    \item Desenvolver uma interface intuitiva para professores, que permita o envio de planos de aula e o acompanhamento do progresso dos alunos, além de incorporar um módulo de feedback da IAG capaz de sugerir novos focos de personalização com base nos dados de engajamento dos alunos.\\
\end{itemize}
