A disfunção neurocomportamental refere-se sobretudo por desatenção, inquietude e impulsividade, essas representam características associadas ao Transtorno de Déficit de Atenção com Hiperatividade (TDAH). Tende a surgir durante a infância, prejudicando o processo de aprendizagem e posteriormente perdurar na fase adulta, afetando, também, a vida social e profissional \cite{Einstein2024}. Diante de tal conjuntura, a Organização das Nações Unidas (ONU) realizou um apelo global a fim de promover o desenvolvimento sustentável através da agenda 2030, um acordo global que visa promover a paz e a proteção do planeta por meio de atividades sustentáveis em diversas áreas, como erradicação da pobreza, redução da fome, igualdade de gênero e combate às mudanças climáticas. O terceiro Objetivo de Desenvolvimento Sustentável (ODS) tem como finalidade, justamente, garantir a saúde e promover o bem-estar para todas as pessoas, independentemente da idade. Ademais, a quarta ODS promete educação acessível e de alta qualidade, estabelecendo oportunidades de aprendizagem ao longo da vida para todos, estão então, ambos os objetivos intrinsecamente relacionados à problemática da disfunção neurocomportamental.

O Ministério da Saúde , com embasamento de dados da Associação Brasileira de Déficit de Atenção (ABDA), indica que o índice de casos de TDAH oscila entre 5\% e 8\% a nível mundial \cite{Brasil2022}. Calcula-se, também, que 70\% das crianças com o transtorno apresentam outra condição associada, sendo as principais depressão, ansiedade e o transtorno opositivo desafiador (TOD) \cite{Zolin2024}.

No Brasil, a taxa de identificação desse transtorno, o TDAH, varia de 1,8\% a 5,8\%. Historicamente, essa doença foi ignorada ou diagnosticada de forma imprecisa, com muitas pessoas portando-a sem ter consciência disso. No contexto hodierno, os avanços em psiquiatria e medicina, associados à maior conscientização, estão possibilitando que mais indivíduos sejam devidamente reconhecidos \cite{Senado2023}. Por conseguinte, conquanto o número de diagnósticos tenha aumentado, isso não quer dizer que o transtorno esteja se disseminando, mas sim que estamos melhor equipados para reconhecê-lo.

Os principais sintomas do TDAH se concentram de maneira mais prejudicial em crianças e adolescentes, visto que é o período em que são identificados primordialmente. Dentro do conjunto de sinais predominantes do transtorno estão: desatenção, identificada por não prestar atenção em pequenos detalhes ou cometer muitos erros por descuido, parecer não consciente quando alguém lhe dirige a palavra e ter dificuldade em manter atenção em tarefas \cite{PequenoPrincipe2023}; hiperatividade e impulsividade, reconhecidas por inquietação motora e tendência a agir de forma precipitada \cite{PequenoPrincipe2023}.

Apesar da inadvertência causada pelo distúrbio, muitas pessoas vivenciam momentos de hiperfoco. Trata-se de um envolvimento profundo em atividades ou interesses específicos. Esses momentos de concentração podem ser considerados uma característica contraditória do TDAH. Durante eles, os indivíduos são capazes de focar de modo absoluto em tarefas, bloqueando efetivamente qualquer outro desvio de atenção, é como se eles mergulhassem em um túnel mental onde nada mais importa além da tarefa em questão \cite{Bernardes2022}.

Dentro da sala de aula, levando em consideração os sintomas apresentados nos parágrafos anteriores, quando não diagnosticado, as relações com professores e demais profissionais da escola podem ser muito problemáticas, já que é comum que essas crianças sejam confundidas com alunos que apresentam condutas consideradas inadequadas, como inatenção constante, impulsividade e dificuldade de seguir regras, ou simplesmente com pura falta de interesse nos estudos. Desse modo, a falta de atenção, seja por hiperatividade ou desatenção, é prejudicial para o convívio escolar, comprometendo a assimilação dos conteúdos e levando o estudante a acumular deficiências de aprendizagem ao longo do tempo. A ausência de uma educação inclusiva e de qualidade para crianças com TDAH intensifica esse cenário, comprometendo seu desenvolvimento cognitivo, emocional e social, configurando-se como uma barreira significativa para a promoção do bem-estar (ODS 3) e para a efetivação de uma educação equitativa e de qualidade (ODS 4). Nesse sentido, a Declaração de Salamanca, uma ação da Organização das Nações Unidas voltada à definição de valores e diretrizes para a educação especial, estabelece como princípio fundamental que todas as crianças devem aprender juntas, independentemente de quaisquer dificuldades ou diferenças que possam ter. No entanto, \cite[p.10]{Santos2023} aponta que, para que alunos com tais condições possam alcançar o nível de aprendizado dos demais, os profissionais de ensino,  mesmo quando não são capacitados para lidar com crianças com TDAH, precisam se desdobrar para oferecer um apoio mais individualizado, garantindo que o estudante não fique para trás.

Em razão disso, esses educadores precisam adotar métodos pedagógicos diferenciados para potencializar a aprendizagem desses alunos, como a variação de rotinas e a inclusão de atividades diversificadas. Incentivar a prática e repetição também é importante, visto que essas crianças podem ter dificuldades para assimilar sequências e manter o foco. Pensando nisso, também é essencial sempre dividir as tarefas em partes bem definidas de modo a prevenir confusões. Além disso, é recomendável fornecer um reforço positivo cada vez que ela realizar uma tarefa com êxito, por meio de elogios ou recompensas \cite{SM2025}.

Quando a criança apresenta hiperfoco, essa característica pode ser aproveitada como um recurso pedagógico valioso. Embora grande parte das pesquisas sobre o uso educacional do hiperfoco seja voltada a estudantes com Transtorno do Espectro Autista (TEA), os resultados observados nessas investigações oferecem subsídios relevantes para o contexto do TDAH. Os achados descritos por \cite{Souza2024} indicam que a exploração dos interesses específicos dos alunos promove maior engajamento, concentração e satisfação em atividades de aprendizagem. Além disso, o uso do hiperfoco como estratégia pedagógica favorece a inclusão e o desenvolvimento de competências cognitivas e sociais. Assim, ao transpor esses princípios para o campo do TDAH, entende-se que adaptar os conteúdos escolares aos interesses e focos de atenção das crianças pode transformar uma característica frequentemente vista como limitante em uma ferramenta que potencializa o aprendizado. Nesse sentido, é essencial que os docentes, em diálogo constante com as famílias, reconheçam os momentos em que o hiperfoco se manifesta e os direcionem para atividades que demandem concentração, resolução de problemas ou produção criativa. Por exemplo, se um aluno demonstra hiperfoco em atividades artísticas, o professor pode incentivá-lo a criar ilustrações, maquetes ou projetos visuais relacionados ao conteúdo escolar, transformando seu interesse intenso em aprendizado significativo. Entretanto, quando mal administrado, o hiperfoco pode gerar atrasos, desorganização e dificuldades de relacionamento \cite{Cellera2025}.


Consequentemente,  é importante que esses profissionais saibam lidar com as diferenças de aprendizagem de cada criança, tendo hiperfoco ou não, para apoiar e elevar o desenvolvimento dela, ajudando a integrá-la com os demais colegas de classe. Como resultado desse conjunto de responsabilidades, é inevitável o desgaste desses docentes, pois uma mesma sala pode reunir várias crianças com déficit de atenção, exigindo do educador um esforço ainda maior para atender a cada uma delas. 

Portanto, a presente pesquisa tem como objetivo auxiliar o professor, através de uma aplicação móvel de suporte educacional que modifique o plano de aula por algo mais lúdico utilizando o hiperfoco, conforme mencionado anteriormente. O sistema empregará Inteligência Artificial Generativa (IAG) que será utilizada para traduzir/transformar o conteúdo textual do plano de aula em prompts para a geração de elementos lúdicos visuais/narrativos (Ex: gamificação de desafios ou ilustrações temáticas) que se alinhem ao interesse de hiperfoco da criança. Para controle de uso, o aplicativo também possuirá um cronômetro que será utilizado para fazer controle do tempo em que a criança passa no aplicativo. Além disso, apresentará a evolução do aluno com base nas entregas de atividades. A proposta é conectar o professor, pai e a criança com TDAH, permitindo que todos tenham acesso ao histórico de desempenho e possam fazer um acompanhamento mais próximo,  trabalhando suas principais dificuldades. Dessa forma, a iniciativa alinha-se diretamente com os Objetivos de Desenvolvimento Sustentável mencionados. Promovendo uma educação de qualidade inclusiva (ODS 4) ao mesmo tempo em que zela pelo bem-estar e saúde mental da criança (ODS 3). 

Diante desse cenário, a questão de pesquisa é: Como a aplicação de IA Generativa pode adaptar planos de aula de maneira lúdica, explorando o hiperfoco de crianças com TDAH, e qual o impacto mensurável dessa personalização no aprimoramento da atenção e retenção do conteúdo no ensino fundamental I?